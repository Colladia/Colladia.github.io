\documentclass[11pt]{beamer}
\usepackage[T1]{fontenc}
\usepackage[utf8]{inputenc}
\usepackage{lmodern}
\usepackage[francais]{babel}

\usepackage{textpos}

\usepackage{graphicx}
\usepackage{tabularx}
\usepackage{lastpage}
\usepackage{xcolor}
\hypersetup{urlcolor=blue, linkcolor=blue, citecolor=black}
\usepackage{nameref}
\usepackage{amsmath} % equation align
\usepackage{amsfonts}
\usepackage{amssymb}
\usepackage{verbatim}
\usepackage{tikz}

\makeatletter

% create beamer ball commands
\newcommand\beamerball{%
	\parbox[t]{10pt}{\raisebox{0.2pt}{\beamer@usesphere{item}{bigsphere}}}}

\newcommand\tikzball[1]{%
	\parbox[t]{10pt}{%
		\tikz[baseline=(char.base)]{%
			\node[circle,ball color=blue, shade, color=white,inner sep=1.2pt] (char) {\tiny #1};
		}
	}
}

% create new list for increased depth
\usepackage{enumitem}
\renewlist{itemize}{itemize}{6}
\setlist[itemize]{label=\beamerball, labelsep=0pt, leftmargin=2em, itemsep=\parskip}
\setlist[itemize, 1]{leftmargin=1.2em}

\renewlist{enumerate}{enumerate}{6}
\setlist[enumerate]{label=\protect\tikzball{\arabic*}, labelsep=0pt, leftmargin=1.4em, itemsep=\parskip}

% reduce left and right text margin
\usepackage{geometry}
\setbeamersize{text margin left=2em}
\setbeamersize{text margin right=1em}

\newcommand*{\currentname}{\@currentlabelname}

%--------------%
% beamer theme %
%--------------%

% colors
\definecolor{blue}{HTML}{204A87}
\definecolor{darkblue}{HTML}{183866}
\definecolor{lightblue}{HTML}{275BA6}
\definecolor{grey}{HTML}{D3D3D3}

% outer theme
% \useoutertheme[footline=empty, subsection=false]{smoothbars}
% \useoutertheme{infolines}
\useoutertheme{default}

% inner theme
\useinnertheme{rounded}

% color theme
\setbeamercolor{structure}{fg=white, bg=blue}
\setbeamercolor{titlelike}{fg=white, bg=white}
\setbeamercolor{frametitle}{fg=white, bg=blue}
\setbeamercolor{section in head/foot}{fg=white} % miniframe
\setbeamercolor{section in toc}{fg=black, bg=white}
\setbeamercolor{subsection in toc}{fg=black, bg=white}
\setbeamercolor{subsubsection in toc}{fg=black, bg=white}
\setbeamercolor{item in toc}{fg=red, bg=green}
\setbeamercolor{item}{fg=blue, bg=blue}
\setbeamercolor{block body}{bg=grey}
\setbeamercolor{palette primary}{bg=lightblue, fg=white}
\setbeamercolor{palette secondary}{bg=blue, fg=white}
\setbeamercolor{palette tertiary}{bg=darkblue, fg=white}

% font theme
\setbeamerfont{section in toc}{size=\small}
\setbeamerfont{subsection in toc}{size=\small}
\setbeamerfont{subsubsection in toc}{size=\small}
\setbeamerfont{author}{size=\footnotesize}
\setbeamerfont{date}{size=\small}

% navigation
\setbeamertemplate{navigation symbols}{}

% other
% \setbeamertemplate{blocks}[default] % squared blocks

\makeatother


% title info
% \title{Projet IA04 - NF28}
% \subtitle{Colladia, éditeur de diagramme collaboratif}
\author{Jean Vintache\\Florian Impellettieri\\Charles Menier\\Marouane Hammi\\Adrien Jacquet}
% \author[short]{author}
\date{\today}
% \titlegraphic{\vspace{-2.25cm}\includegraphics[height=4cm]{img/gliese581_2.jpg}}
\logo{ % logo on titlepage only
	\ifnum \insertpagenumber=1
		\vspace{-0.2cm}\includegraphics[scale=0.1]{img/utc-logo.jpg}
	\fi
}

% \AtBeginSection[]
% {
%    \begin{frame}%[allowframebreaks] % multiple frames
% 	   \frametitle{Sommaire}
% 	   \tableofcontents[currentsection]
%    \end{frame}
% }


\begin{document}
\title{
		\hspace*{-.55cm}
		\begin{tikzpicture}
		\node [inner sep=0pt] at (0,0) {\includegraphics[width=\textwidth]{img/colladia-banner}};
		\draw [white, rounded corners=\ClipSep, line width=\ClipSep] 
				(current bounding box.north west) -- 
				(current bounding box.north east) --
				(current bounding box.south east) --
				(current bounding box.south west) -- cycle
				;
		\end{tikzpicture}
}

\small
\begin{frame}
	\vspace{-0.75cm}
	\titlepage
\end{frame}

%\begin{frame}%[allowframebreaks] % multiple frames
%	\frametitle{Sommaire}
%	\tableofcontents
%\end{frame}

\section{Présentation du projet}
\begin{frame}
	\frametitle{\currentname}
	\begin{itemize}
		\item Éditeur de diagramme collaboratif pour Android
		\\~\\
		\item Client (Contexte NF28)
		\begin{itemize}
		  \item Java Android natif
			\item Communication avec le serveur via des requêtes HTTP REST
			\item Synchronisation par des requêtes GET périodiques
		\end{itemize}
		~\\
		\item Serveur (Contexte IA04)
		\begin{itemize}
			\item Système multi-agent JADE
			\item Sauvegarde et synchronise l'état des diagrammes entre les différents clients
			\item Implémente des fonctions d'ajout, modification, suppression et auto-positionnement d'éléments
		\end{itemize}
	\end{itemize}
\end{frame}

\section{Organisation des vues}
\begin{frame}
	\frametitle{\currentname}
	\begin{center}
		\includegraphics[width=\textwidth]{img/screens}
	\end{center}
\end{frame}

\section{Fonctionnalités}
\begin{frame}
	\frametitle{\currentname}
\begin{itemize}
  \item 
\end{itemize}
\end{frame}

\section{Démonstration}
\begin{frame}
	\frametitle{\currentname}
	\begin{itemize}
	\item \href{https://www.youtube.com/watch?v=1xX70e88ro8}{Vidéo de démonstration}
  \vspace*{\fill}
	\item Colladia est une organisation github ( \href{https://github.com/Colladia/}{github.com/Colladia} )
	\end{itemize}
\end{frame}

\section{Améliorations envisagées}
\begin{frame}
	\frametitle{\currentname}
  \begin{itemize}
    \item Gestion des utilisateurs connectés
      \begin{itemize}
        \item notification de connexions
        \item chatter
        \item attirer l'attention
      \end{itemize}
    \item 
    \item 
    \item 
  \end{itemize}
\end{frame}

\section{Remarques, questions}
\begin{frame}
  \frametitle{\currentname}
  \begin{center}
  ?
%  \includegraphics[maxwidth]{img/thats-a-question}
  \end{center}
\end{frame}

\end{document}
